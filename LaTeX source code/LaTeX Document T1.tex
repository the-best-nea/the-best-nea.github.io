\documentclass[11pt]{report}
\usepackage[margin=1in]{geometry}
\usepackage[table,xcdraw]{xcolor}
\usepackage{tabularx}
\renewcommand\tabularxcolumn[1]{m{#1}}% for vertical centering text in X column
\renewcommand{\arraystretch}{2} % for cell padding
\newcolumntype{b}{>{\hsize=.45\hsize}X}
\newcolumntype{s}{>{\hsize=.1\hsize \raggedright\arraybackslash}X}
\newcolumntype{S}{>{\hsize=.07\hsize \raggedright\arraybackslash}X}
\newcolumntype{B}{>{\hsize=.23\hsize}X}
\usepackage{color}
\usepackage{animate}
\usepackage{graphicx}
\graphicspath{ {./images/} }
\usepackage{hyperref}
\hypersetup{
    colorlinks=true,
    linkcolor=black,
    urlcolor=blue,
    linktoc=all
}
\setcounter{tocdepth}{3}
\setcounter{secnumdepth}{4}

\newcommand*{\thead}[1]{\multicolumn{1}{c}{\bfseries #1}}
\newcommand{\comment}[1]{}

% Title Page
\title{\textbf{Pluto Rover - Autonomous Navigation}}
\author{Pallav Hingu}
\date{}

\begin{document}

\maketitle
\tableofcontents

\newpage
\chapter{Analysis}

\section{Description of the Project}

Space Exploration has advanced massively in recent years. Large companies such as NASA and SpaceX are taking large steps forward to explore and colonize new planets such as Mars. We know a lot about the planets in our solar system: We have pictures, samples and critical data about their atmospheres. This has opened a gateway allowing us access to further resources from the planets in our solar system. But, there is one "planet" that we have yet to explore in detail; a "planet" where no man-made object has never landed: Pluto. Having been recently described on Google as: \emph{"Our favourite dwarf planet since 2006"}, we have yet to discover its surface and internal composition. To continue our research of the planets in our solar system, we will need to \textbf{send a rover on a research and exploration mission to Pluto.} \\

As Pluto is extremely far away from the Earth, there are many obstacles to sending a rover to Pluto. Whilst the most obvious is transportation, NASA have already proved that they are capable of doing this using their "New Horizons" interplanetary space probe launched as part of the "New Frontiers" program. But, after getting to Pluto, there are many challenges that a rover can face. Firstly, due to the large distance between Earth and Pluto, there is a considerable delay in receiving signals, and there is also a lot of data loss. This means that the rover cannot easily be controlled remotely as there would be a large delay in sending instructions as well as receiving camera/sensor feed, which means that the camera may not represent the actual situation of the rover when the feed is received on Earth. As this poses a large threat to the condition of the rover, the best approach to address the safety of the robot is to install an autonomous navigation software, which avoids all obstacles in its path and uses other available sensors on the rover to get more information on the surroundings.\\

The information received using these sensors will be stored on the rover as well as sent to the nearest orbiting satellite, which would have been used to transport the rover as well. This satellite will then also store the data received from the rover's sensors and then will broadcast these results to Earth, allowing it to be analysed.\\

I will design a software with a friendly user interface (which will be on the companion app) that presents live information from the rover on a remote device, which can be on a satellite or on Earth to monitor the rover and receive crucial data. The rover is on Pluto to detect the presence and location of metals on the surface. My software will utilize IoT protocols to present the output of the metal detector. It will also use a built-in IR camera to perform autonomous driving. A smart algorithm will also be used to generate the relative location of the rover on Pluto's surface according to its set landing location and landing orientation. This will allow the rover's position to be estimated, as GPS cannot be utilised on Pluto. \\
 
The aim of this mission is to detect if there are metals near the surface of Pluto, if yes, then to identify and record where.\\

\section{Computational Methods}

I think that this problem lends itself well to computational methods such as abstraction and decomposition for a variety of reasons. The solution will be a software that controls the Pluto rover on a hardware level, allowing it full autonomous control as well as manual, using the rover's cameras and sensors. This will simply need to run on a computer as that will be the medium on which the images from the camera will be processes and the next course of action for the rover will be determined. There is no alternative solution for this problem that will not require a computer, due to the context of the problem, and the environment in which the solution will need to function (the harsh climate of Pluto).

\subsection{Methods Used}

Some of the computational methods used for the solution are:

\begin{itemize}
	
	
	\item{\textbf{Problem Decomposition}}
	\\This large problem can be decomposed into many smaller steps. These steps can be thought to be the 			following:
	\begin{enumerate}
		\item{Use the rover's camera system to get some RAW data about the surroundings}
		\item{Add threshold to data to limit the amount of data being passed through the program}
		\item{Plot an image representation of the RAW data}
		\item{Use image processing to detect the obstacles in the image}
		\item{Use a smart algorithm to determine the action needed to avoid obstacles}
		\item{Overwrite algorithm-determined movement with manual instructions from companion app if manual mode is enabled.}
		\item{Send instructions to a driver board (allowing hardware control to motors)}
		\item{Use an algorithm on the driver board to control the motors}
		\item{Send accurate movement of the rover to the companion app}
		\item{Calculate vector location of rover using its movement(companion app)}
		\item{Get data from the metal detector}
		\item{Send data to companion app}
		\item{Write vector location and metal detector data to local file (companion app)}
	\end{enumerate}
	After these steps are completed, the problem can be said to be solved. The steps will allow the rover to 			react to obstacles in its surroundings, allowing for autonomous navigation. Also, the steps allow the 			rover to utilise a metal detector and send its data to a server for analysis, meaning that we can get 			more information about the surroundings of the rover and the surface of Pluto. The vector location 			calculation will also allow for the rover to be located using no other hardware components, using only 			the velocity and angular velocity of the rover.
	
	\item{\textbf{Problem Recognition}}	
	\\The overall and apparat problem is autonomous navigation through an extreme terrain. But, the underlying 		and primary problem is the detection of obstacles and their location. If the obstacles can be located, 			this information can be passed through a set algorithm that will allow a motor response.
	
	\item{\textbf{Thinking Procedurally - Divide and Conquer}}
	\\All the decomposed steps above allow for the problem to be divided into smaller problems, which are easier 			to solve than the large overarching problem. Creating functions/modules to solve the smaller problems, 			and them combining them all into a larger modular program is an example of divide and conquer. 
	
	\item{\textbf{Abstraction}}
	\\Abstraction plays a large role in ensuring that the program responds to input data immediately. The 			incoming data stream from the camera sensor contains a lot of information, most of which is irrelevant 			for this application. So, it is vital to utilise abstraction and extract useful information only from 			the camera's input data. In this case, we would add a threshold so that the data only represents 			obstacles a certain distance from the camera. This allows us to reduce the amount of information being 			fed into the program, decreasing its response time and increasing its speed. The obstacles that are 		detected by the camera but are further away from the camera are removed as they do not need to be 			considered yet; these obstacles do not pose a risk of collisions yet.   

	\item{\textbf{Logic Modelling}}
	\\Logic modelling will allow me to illustrate how the program must work to fulfil the system requirements. 			It will also be helpful to depict the actual program's activities compared to its intended needs, 			allowing me to measure how successful the program is at fulfilling it's function and solving the 			problem. As this will help with the application evaluation, it will be useful when improving the program. It ensures that all the requirements are fulfilled well. 

\end{itemize}

% Code Cleaned up from this point. Will do above later. ---------------------------------^

\newpage
\section{Stakeholders}

	The ideal stakeholders for this project would include the large space research firms and their representatives. They would be the clients that would use this software and adapt it for space exploration missions. But, as this is not possible right now, the stakeholders for my project will be people who have done research around artificial intelligence, space exploration and robotics. \\

	My first stakeholder will be a family relative who has completed a PhD in artificial intelligence and is current a professor at a university. I will attempt to add some type of artificial intelligence within the software so that it can learn more from its surroundings and adapt to the current situation. The university professor will use my software to create a demonstration for young student visitors during university open days. This will allow the young students to get inspired and learn more about the future application of artificial intelligence. My software will be one of the suitable solutions for the university as it will be free to use and will have a simple concept, GUI and set-up. While many alternative demonstrations of artificial intelligence on robots require a long set-up time, my software will be very simple to get ready and will be reliable so that there are not unexpected surprises on university open days.\\  

	My second stakeholder will be my schools Young Engineers groups who have engineered a prototype Pluto rover for their competition this year. They will use this software to allow their prototype rover to move across a modelled terrain of Pluto. My solution will benefit this group highly as it will allow them to test their prototype, while also testing my software in the most realistic situation that I have access to. The solution will be most appropriate to this group's needs as it will provide them with a free software that is specifically designed for their rover's hardware and is optimised for the computational power that is available to them. Other solutions would either cost a lot of money or time and would need much more computation power, making them very expensive and complex. \\


\newpage
\section{Research}
	\subsection{Existing Solutions}

		As this is not a commonly commercially developed program, there are very few re-viewable examples available online. It is also very specific to the context, which may be a reason why the program is uncommon. But, if we widen the scope and look for software that are not specifically designed for this application but can be used in this context, we can find more programs online. \\\\Examples:

		\begin{itemize}
			\item{\textbf{RTAB-Map with ROS}}
				\\\\
					\begin{frame}{}
						\animategraphics[loop,autoplay,width=\linewidth]{12}{./images/rtabmap/animate_}{0}{71}
					\end{frame}
				\\
				\emph{\textbf{Overview}}
				\\\\RTAB-Map (Real-Time Appearance Based Mapping) is an open-source application which uses different SLAM (Simultaneous localisation and mapping) approaches to generate a 3D dot-cloud of the surroundings of a sensor. It can be installed and used as a stand-alone application to generate 3D dot-clouds based on sensor inputs, or can be integrated into ROS (Robot Operating System) to allow the application to control hardware according to the 3D dot-cloud generated by a camera/sensor on a robot. It produces a 3d Map of the surrounds and uses localisation to detect the robot's location in relation to the 3d generated map. it then uses this to detect obstacles and find the most efficient way around the map to a given goal location.\\\\\\ROS, which is essential in this application, is a robotics middle-ware and allows the RTAB-Map here to publish real-time messages to the appropriate nodes. In ROS, a node is a process that performs computation. Nodes are combined together in a graph and can communicate with each other to perform certain tasks. RTAB-Map, when integrated well with ROS, publishes messages to many external nodes, which communicate with hardware drivers in the system to move the robot according to RTAB-Map's instructions. As RTAB-Map can get a large image of the surroundings, it uses a smart algorithm to conduct path finding, meaning that it calculates the most efficient route to reach a destination while avoiding obstacles. \\\\As this is a complicated process, it requires a lot of advanced hardware and a high amount of processing power. ROS is also extremely difficult to use, as well as install correctly. ROS is used for many complex robots, like the Robonaut 2 aboard the International Space Station.
				\\\\
				The Graphical User Interface(GUI) that it uses displays lots of information, such as the 3d Dot-map created of the surroundings in real-time, the raw images that are being captured, and the calculation of odometry (the relative location of the vehicle). This is a simple but effective and advanced user interface that provides a lot of data that can be useful for environmental analysis, but is not hard to use. As this application is not specific to a problem, the GUI can be adjusted to display less or more information, or different information by tweaking the settings. This allows the application to be adapted to many different situations, making it ideal even for an autonomous Pluto Rover.  
				\\
				\\\emph{\textbf{Parts that I can apply to my solution}}
				\\\\
				From this application, there are many things that I can use in my application:
					\begin{itemize}
						\item{\underline{\emph{Display Odometry}}} - the application displays the odometry (relative location) of the vehicle on the GUI. This will allow the relative location of the rover to be displayed on the GUI so that the Rover can be located.
						\item{\underline{\emph{Hardware Control}}} - the software can control hardware directly by converting the output into electrical signals for the components. This will allow the Rover's motors to be controlled by my application directly.
						\item{\underline{\emph{Navigation}}} - this application can determine movement to navigate around a map with a set objective. This will be useful to move the Rover around the surface of Pluto with an objective to detect metal.
						\item{\underline{\emph{GUI settings}}} - the use of GUI settings allows the displayed information on the GUI to be adjusted according to user preference or need. This can also be helpful in low resource scenarios as displaying less information means that the program runs smoother and faster. This could be helpful on the Rover as there are limited computational resources and so the adjustment of the GUI to the scenario may be helpful in improving performance.  
					\end{itemize}
				\newpage
	
			\item{\textbf{OpenVSLAM}}
			\\\\
				\begin{frame}{}
			  		\animategraphics[loop,autoplay,width=\linewidth]{12}{./images/OpenVSLAM/animate_}{0}{70}
				\end{frame}
			\\
			\emph{\textbf{Overview}}
			\\\\
			OpenVSLAM is a monocular, stereo and RGB-D visual SLAM system similar to RTAB-Map. It is a software that uses pattern recognition to detect movement and estimate current location of the object, its odometry. But, the  large advantage with this application is that it is compatible with many different cameras and so can be used for a variety of different projects. This system is also completely modular as it is designed by encapsulating several functions in separated components with APIs. \\
	
			When comparing OpenVSLAM with RTAB-Map, we can see that RTAB-Map displays a lot more data. This is because it creates a dot-cloud with as much detail as possible to generate an accurate computational representation of its surroundings. While this data may be useful for many applications, it is not essential for obstacle detection and autonomous navigation. On then other hand, OpenVSLAM generates a simple but informative 2D birds eye map of the movement of the vehicle. It also creates a simpler dot-cloud by only detecting the presence of obstacles and not their shape, size or colour. This allows it to use minimal relevant data to navigate, which is optimal as it reduces the amount of data being processing, reducing processing time, latency and hardware demand. \\
	
			The minimal approach on this monocular SLAM software makes it more suited for our application than RTAB-Map as we have limited hardware, and an overload of unnecessary data may product uncertainty in the reliability of the software. \\
			\newpage
			\emph{\textbf{Parts that I can apply to my solution}}
			\\\\There are a few things from OpenVSLAM that I can apply to my solution:
				\begin{itemize}
					\item{\underline{\emph{Input Abstraction}}} - the ability to abstract the relevant information form the 		input device to suit the need of the software. This allows the software to be more efficient as well as 		reliable.
					\item{\underline{\emph{GUI Simplicity}}} - the GUI for OpenVSLAM is very simple and effective as it 		displays essential data making it easy to read, ensuring that there is not an overload of data being 			displayed to cause software lag as well as user confusion. 
				\end{itemize}
	
	
		\end{itemize}
		\newpage

	\subsection{Primary Research}
		\subsubsection{Interview Questions}

			To ensure that an interview with the stakeholders is as effective as possible, the questions to be asked will require planning so that all required information related to the solution can be communicated. \\\\
			I will first interview the \textbf{\emph{Artificial Intelligence Professor}} with the following questions.\\

			\underline{Background Questions:}
				\begin{enumerate}
					\item{\emph{How important do you think artificial intelligence to be for the Pluto rover?}}
					\item{\emph{In this case, what can artificial intelligence do that a standard algorithm cannot?}}
					\item{\emph{How will adding artificial intelligence impact the hardware utilisation of the program?}}
				\end{enumerate}

			\underline{Questions for client requirements:}
				\begin{enumerate}
					\item{\emph{What are your expectations for the GUI?}}
					\item{\emph{What hardware would you be able to use for running this software?}}
					\item{\emph{What should you be able to control remotely?}}
					\item{\emph{Would you prefer using a companion app on a computer or wiring up the embedded computer?}}
					\item{\emph{Is there anything else you would like to add?}}
				\end{enumerate}
			Apart from the above question, I may ask follow up questions or ask them to elaborate on certain answers.\\
			\\
			After completing the interview with the Professor, I will interview the \textbf{\emph{Young Engineers group}} with the following questions.\\


			\underline{Questions for client requirements:}
				\begin{enumerate}
					\item{\emph{What are your expectations for the GUI?}}
					\item{\emph{What hardware would you be able to use for running this software?}}
					\item{\emph{What should you be able to control remotely?}}
					\item{\emph{Would you prefer using a companion app on a computer over wiring up the embedded computer?}}
					\item{\emph{Is there anything else you would like to add?}}
				\end{enumerate}
			These questions are the same as the client requirement questions from the Artificial Intelligence Professor, allowing us to compare and create a collective requirements list to suit both stakeholders.\\\\
			I have included some background questions for the Professor so that I can use the answers when attempting to implementing AI into my program.  

		\subsubsection{Interview}
			\underline{Artificial Intelligence Professor:}\\

			Background Questions:
				\begin{enumerate}
					\item{\emph{\textbf{How important do you think artificial intelligence to be for the Pluto rover?}}}\\\\
					\emph{"I think that the use of artificial intelligence is extremely important in a rover, especially if it made for Pluto as the distance between the user and the rover will be very far. Therefore, we will need something to control the rover locally without relying on a human, and so artificial intelligence will be needed."}
					\item{\emph{\textbf{In this case, what can artificial intelligence do that a standard algorithm cannot?}}}\\\\
					\emph{"Well there are many things that an artificial intelligence model can be used for in this scenario. For example, it can be used to move the rover by observing its surroundings, it can be used to predict and recognize hazards. Unlike standard algorithms, it can learn more from the information it is observing, allowing it to perform better in unaccounted situations."}
					\item{\emph{\textbf{How will adding artificial intelligence impact the hardware utilisation of the program?}}}\\\\
					\emph{"Well, there are many different ways of implementing artificial intelligence, so it does depend on your chosen method. But, in general, AI utilises more hardware than a standard algorithm, especially if it is constantly learning from it's environment. So, I would say that adding AI to the program will increase its hardware requirements quite dramatically."}
				\end{enumerate}
			~\\
			Questions for client requirements:
				\begin{enumerate}
					\item{\emph{\textbf{What are your expectations for the GUI?}}}\\\\
					\emph{"In this case, it would be most effective to use a simple GUI that allows control over all vital hardware features that are accessible using the software. I would also expect to see some sort of statistics on the current state of hardware or the output of mounted sensors."}
					\item{\emph{\textbf{What hardware would you be able to use for running this software?}}}\\\\
					\emph{"For demonstrations with this software,  I would have access to a computer and possibly the model rover needed for the physical demonstration. It would be ideal if I could use any computer to control the rover."}
					\item{\emph{\textbf{What should you be able to control remotely?}}}\\\\
					\emph{"As the software is designed for long range control, I would expect options for manual control for as many things as possible. there are many things that can go wrong and so the user should have the choice to control all hardware possible remotely, as well as the movement of the rover obviously."}
					\item{\emph{\textbf{Would you prefer using a companion app on a computer or wiring up the embedded computer?}}}\\\\
					\emph{"It would definitely be easier to use a companion app, as well as much more professional. Wiring would increase the chances of errors and make everything much more fiddly during demonstrations, and so I would definitely prefer a companion app. But, the companion app should be easy to use and should be reliable, having quick setup times as well to make it robust."}
					\item{\emph{\textbf{Is there anything else you would like to add?}}}\\\\
					\emph{"I think I have said everything I would like to. Thank you."}
				\end{enumerate}
			~\\
			\underline{Young Engineers Group:}

				\begin{enumerate}
					\item{\emph{\textbf{What are your expectations for the GUI?}}}\\\\
					\emph{"A clean user interface with complex functionality allowing full control of the rover. We vision the GUI with a clean look presenting the location of the rover and a few buttons that allow full control of the rover."}
					\item{\emph{\textbf{What hardware would you be able to use for running this software?}}}\\\\
					\emph{"To run the embedded app, we have a Raspberry Pi 3 that is built into our model rover. Alongside, we also have an Arduino PCD that is hardwired to all rover sensors and motors. The Arduino can also communicate with the Raspberry Pi. If needed, we also have access to a remote laptop. Regarding the rover's sensors, it has a Kinect 360 RGB-D camera along with a magnetic coil for metal detection and a receiver for a RC transmitter. The Rover also has 4 DC motors that can be used for movement. This is the hardware that we would ideally run the application on."}
					\item{\emph{\textbf{What should you be able to control remotely?}}}\\\\
					\emph{"It would be nice to be able to control the rover's movement remotely to control the rover as required during testing and demonstrations. We should also be able to switch quickly between autonomous and manual control, which would helpful during presentations of the rover at events."}
					\item{\emph{\textbf{Would you prefer using a companion app on a computer over wiring up the embedded computer?}}}\\\\
					\emph{"It would be helpful to have the option use a companion app but we would also want to be able to control the rover with a RC transmitter so that it can be manually controlled without having do download specific software."}
					\item{\emph{\textbf{Is there anything else you would like to add?}}}\\\\
					\emph{"One thing we would like to add is that the Kinect sensor can see objects that are not directly in front of the rover. This may need to be considered during development."}
				\end{enumerate}

		\subsubsection{Interview Analysis}

			During the interview with the University Professor, I asked a few background questions to allow me to expand my knowledge and understand the effects and impacts when attempting to add artificial intelligence to my program. From these background questions, I have learnt that adding artificial intelligence to the software would be extremely helpful for my purpose but it can have many obstacles such as higher hardware requirements.\\
			\\
			From the other questions that I asked the University Professor and the Young Engineers group, I learnt that the GUI would need to be basic in appearance but should have the ability to control all the hardware possible. It should also present statistics about the current state of the rover along with the location of the rover. Ideally, it should also allow the rover to be switched between manual and autonomous mode quickly. The program should also be extremely efficient, using very low processing power as it needs to be run on a single board computer like the Raspberry Pi 3. There are also many hardware units hardwired to this single board computer and so the program needs to be able to control those units appropriately. Remotely, both stakeholders would prefer being able to control all hardware possible, especially the movement of the rover. The software should also include a professional and easy to use companion app, as preferred by both stakeholders. While being able to control the rover using a companion app, the Young Engineers group would also want to be able to control the rover using a remote transmitter that has a hardwired receiver. At the end of the interview with the Young Engineers group, they have also pointed out that the camera sensor has a wide perspective and so can see objects that are not directly in front of the rover, meaning that the software has to differentiate between objects that are directly in front and so are an obstacle, and objects that are not in the rover's path. 

\section{Essential Features of the Proposed Solution (System Goals)}
	From the research of existing software and interviews, a system can be designed as a solution to the initial problem. The solution proposed for this problem would have the following features:
	~\\\\\\
	\begin{tabularx}{1\textwidth} { 
		  | >{\centering\arraybackslash}X 
		  | >{\centering\arraybackslash}X 
		  | >{\centering\arraybackslash}X | }
			 \hline
			 	\textbf{Feature} & \textbf{Description} & \textbf{Purpose} \\
			 \hline
			 	Map  & Display the location  & The stakeholder will be able to view the relative location of the rover to the starting position.  \\
			 \hline
			 	Menu & Buttons for control & The menu will be a collection of controls that will allow the hardware of the rover to be controlled. \\
			\hline
	\end{tabularx}
	\begin{tabularx}{1\textwidth} { 
		  | >{\centering\arraybackslash}X 
		  | >{\centering\arraybackslash}X 
		  | >{\centering\arraybackslash}X | }
			  \hline
			 	Data Abstraction & Control input data & The input data will immediately be abstracted to useful data so that less processing is needed in the program, decreasing hardware utilization. \\
			 \hline
			 	Auto mode & Autonomous navigation of the rover & The user can turn on autonomous mode to control the rover's movement using the internal algorithm. \\
			 \hline
			 	Manual mode & Manual control over rover movement & This is turned on when auto mode is turned of. Manual mode allows the user to have complete control over the rover's movement.\\
			 \hline
			 	Stats Display & Collection of live information about hardware & This will be a small collection of graphs and numbers that allow the live information form the sensors to be monitored. \\
			\hline
	\end{tabularx}
	~\\

\section{Desirable Features of the Proposed Solution}
	\begin{tabularx}{1\textwidth} { 
		  | >{\centering\arraybackslash}X 
		  | >{\centering\arraybackslash}X 
		  | >{\centering\arraybackslash}X | }
			 \hline
			 	\textbf{Feature} & \textbf{Description} & \textbf{Purpose} \\
			 \hline
			 	AI  & Artificial Intelligence navigation  & The use of a trained Artificial Intelligence model for autonomous navigation.  \\
			 \hline
			 	Data Preserve & Save Essential Data from Rover & Save information of the sensors, co-ordinates of rover and current vital variables in the program. \\
			 \hline
			 	Negative Obstacle Detection & Detect any absence of the ground & Use a more advanced algorithm to detect potholes in front of the rover by detecting the absence of a ground plane. \\
			\hline
	\end{tabularx}


\section{Software and Hardware Requirements}

	\subsection{Software Requirements}
		~\\
		\underline{Embedded Application Requirements:}

		\begin{itemize}
			\item{Operating System with independent requirements:}
			\begin{itemize}
				\item{\textbf{Windows 7+ (x86 or x64) or Windows IoT core}} - \emph{The drivers are only officially available for these releases.}
					\begin{itemize}
						\item{Kinect for Windows SDK} - \emph{To support the Kinect 360 RGB-D Camera}
						\item{Visual Studio 2010 Edition} - \emph{A dependency for Kinect Drivers.}
						\item{Microsoft .NET Framework 4.0} - \emph{A dependency for Kinect Drivers.}
						\item{Java Runtime Environment} - \emph{Required to run my Application as it is Java based.}
						\item{Active Internet Connection} - \emph{To allow communication with companion app.}
					\end{itemize}
		
				\item{\textbf{Ubuntu 18.04}} - \emph{The drivers and software is tested and robust on this specific OS release.}
					\begin{itemize}
						\item{Libfreenect Drivers} - \emph{To support the Kinect 360 RGB-D Camera.}
						\item{Java Runtime Environment} - \emph{Required to run my Application as it is Java based.} 
						\item{A Desktop Environment} - \emph{Needed to interact with the program UI.}
						\begin{itemize}
							\item{Recommended: GNOME Desktop Env.}
						\end{itemize}
						\item{Active Internet Connection} - \emph{To allow communication with companion app.}
					\end{itemize}	
			
				\item{\textbf{Mac OS}} - \emph{Cannot be used as it should not be loaded onto a Raspberry Pi 3 or other single board computers.}
			\end{itemize}
		\end{itemize}
	~\\
	\underline{Companion Application Requirements:}

		\begin{itemize}
			\item{Operating System with independent requirements:}
			\begin{itemize}
				\item{\textbf{Any Windows Release}} - \emph{The JRE can be installed on all Windows releases}
					\begin{itemize}
						\item{Java Runtime Environment (JRE)} - \emph{Required to run my Application as it is Java based.}
						\item{Active Internet Connection} - \emph{To allow communication with embedded app.}
					\end{itemize}
		
				\item{\textbf{Any Ubuntu Release}} - \emph{The JRE can be installed on all Ubuntu releases}
					\begin{itemize}
						\item{Java Runtime Environment (JRE)} - \emph{Required to run my Application as it is Java based.} 
						\item{A Desktop Environment} - \emph{Needed to interact with the program UI.}
						\begin{itemize}
							\item{Recommended: GNOME Desktop Env.}
						\end{itemize}
						\item{Active Internet Connection} - \emph{To allow communication with embedded app.}
					\end{itemize}	
			
				\item{\textbf{Any Mac OS}} - \emph{All macOS releases are supported by JRE}
					\begin{itemize}
						\item{Java Runtime Environment (JRE)} - \emph{Required to run my Application as it is Java based.} 
						\item{Admin Privileges - for installation of JRE}
						\item{Active Internet Connection} - \emph{To allow communication with embedded app.}
					\end{itemize}
				
			\end{itemize}
		\end{itemize}


	\subsection{Hardware Requirements}

		\underline{Embedded Application Requirements:}

		\begin{itemize}
			\item{Kinect 360 RGB-D Camera} - \emph{Needed to detect obstacles and analyse surroundings}
			\item{Computer with a dual-core, 1.2 GHz or faster processor} - \emph{Required for Kinect SDK}
			\item{Windows compatible graphics card that supports DirectX® 9.0c capabilities} - \emph{Required for Kinect SDK on Windows}
			\item{1GB RAM (2-GB RAM recommended)} - \emph{Would be required to run Kinect drivers as well as my application and any other libraries.}
			\item{4 DC motors} - \emph{To act as movement output and move the wheels.}
			\item{Arduino connected with a I2C bus to the Computer} - \emph{To interface with the sensors and motors directly, acting as a driver board.}
		\end{itemize}

		~\\
		\underline{Companion Application Requirements:}
		
		\begin{itemize}
			\item{An Intel-based Mac or any Windows/Ubuntu computer.} - \emph{Requirement for JRE.}
			\item{Minimum 128MB memory available.} - \emph{Requirement for JRE.}
			\item{Cursor Input (Mouse, Trackpad, Touch screen, etc.)} - \emph{needed to use navigate menu and press buttons.}
			\item{Keyboard (optional for running application but required for manual movement control.)} - \emph{Keyboard keys would be used to control the rover manually and so a keyboard will be needed for such use.}
		\end{itemize}

\newpage
\section{Requirements Specification}
	~
	\begin{center}
		\underline{\textbf{\emph{Input Requirements}}}
	\end{center}
	\noindent
	%\begin{tabularx}{\textwidth}{| X | X | X |}
	\begin{tabularx}{\textwidth}{|s|b|b|}
	    \hline
	    \textbf{Number}     & \textbf{Criteria}     & \textbf{Justification}     \\ \hline
	    \textbf{1.1}         & The user should be able to promptly control the movement mode.        & Input from the user will toggle auto and manual mode on or off.          \\ \hline
	    \textbf{1.2}         & The user should be able to send instructions to move using keyboard keys.        & The user will be able to use certain keys from the keyboard to control the rover movement as well as other motors.         \\ \hline
	    \textbf{1.3} & The user will be able to use the menu to hide control buttons and statistics. & This will allow the user to hide objects that they do not need to keep the UI clean and to prevent accidents. \\ \hline
	    \textbf{1.4} & There should be input from the sensors & This will be used to determine the next course of action for the rover, using either an algorithm or from manual control instructions. \\ \hline  
	\end{tabularx}
	
	~\\\begin{center}
		\underline{\textbf{\emph{Processing Requirements}}}
	\end{center}
	\noindent
	%\begin{tabularx}{\textwidth}{| X | X | X |}
	\begin{tabularx}{\textwidth}{|s|b|b|}
	    \hline
	    \textbf{Number}     & \textbf{Criteria}     & \textbf{Justification}     \\ \hline
	    \textbf{2.1}         & The embedded program will use around 300MB of memory.        & This will allow the program to run smoothly on single board computers and slower computers.          \\ \hline
	    \textbf{2.2}         & The program should only account for obstacles within 1m of the sensor.        & This will allow the program to run smoother and use efficient abstraction to make appropriate decisions.         \\ \hline
	    \textbf{2.3} & The program will ignore data of obstacles not on the rover's path. & This will allow the rover to detect only the obstacles directly in from of the rover that pose a direct hazard to the rover. \\ \hline
	    \textbf{2.4} & The data from the sensors will be sent to the companion app within 5 seconds. & This will ensure that the data viewed by the user is an accurate representation of the current situation, allowing the correct actions to be taken in time. \\ \hline
  
	\end{tabularx}
		
	~\\\begin{center}
		\underline{\textbf{\emph{User Interface Requirements}}}
	\end{center}
	\noindent
	%\begin{tabularx}{\textwidth}{| X | X | X |}
	\begin{tabularx}{\textwidth}{|s|b|b|}
	    \hline
	    \textbf{Number}     & \textbf{Criteria}     & \textbf{Justification}     \\ \hline
	    \textbf{3.1}         & The user interface should be clean and simple.        & This will allow the program interface to be suitable for presentations and will only display a small amount of vital information when not in use.          \\ \hline
	    \textbf{3.2}         & The menu will allow full manual control.        & The menu of the user interface should allow the user to toggle on full manual control of the rover where they can override the algorithm decisions manually.         \\ \hline
	    \textbf{3.3} & The UI should present the sensor data. & This will allow the user to analyse the terrain and environment on which the rover is on and make appropriate decisions when manually controlling. \\ \hline
	    \textbf{3.4} & The Embedded Application will be observable. & This will be helpful for debugging as well as presenting software. The Embedded Application can be observed without the companion app. \\ \hline
	    
	\end{tabularx}
	~\\	
		
\section{Success Criteria}

	To ensure that the solution has fulfilled all system requirements, it will need to be tested. I will test the program in 3 different stages. The first stage will test the solution's ability to fulfil all general requirements. The second stage will test the input and output usability of the program. The third stage will test the program's processing functionality and robustness.
	\newpage
	
	\subsection{Stage 1: General Requirements}
	
		\noindent
		%\begin{tabularx}{\textwidth}{| X | X | X |}
		\begin{tabularx}{\textwidth}{|S|B|B|B|B|}
			\hline
			\textbf{Num.}     & \textbf{Criteria}     & \textbf{Desirable}    & \textbf{Acceptable} &  \textbf{Fail}	 \\ \hline
			\textbf{1.1} & The user should be able to promptly control the movement mode.        &  The user can change the movement mode within 3 seconds.  & The user can change the movement mode within 20 seconds & The user cannot change the movement mode.      \\ \hline
			\textbf{1.2} & The user should be able to send instructions to move using keyboard keys.        & The user can send all movement instructions using Keys.	&	The user can only send start and stop.	&   The user cannot send instructions using the keyboard keys.      \\ \hline
			\textbf{1.3} & The user will be able to use the menu to hide control buttons and statistics. & The control buttons will be completely hidden within a menu.	&	Some buttons can be hidden with a menu.	& The menu cannot be hidden and all control buttons are always visible.	 \\ \hline
			\textbf{1.4} & There should be input from the sensors & All sensors send input which can be used. 	&	Most sensors send inputs that are accessible.	&	No sensors on rover input data into the program. \\ \hline  
			
			\textbf{2.1} & The embedded program will use around 300MB of memory.        &  The program uses less than 250MB of RAM  & The program uses between 250MB - 350MB of RAM &   The program uses more than 350 MB of RAM.    \\ \hline
			\textbf{2.2} & The program should only account for obstacles within 0.5m of the sensor.        & The program accounts for all obstacles only within 1m.	&	The program accounts for obstacles within 2m.	&  The program accounts for all visible obstacles.       \\ \hline
			\textbf{2.3} & The program will ignore data of obstacles not on the rover's path. & All obstacles not on the rover's path are ignored.	&	Most obstacles not on rover's path are ignored. 	&	There is no filter for the obstacles not on the rover's path. \\ \hline
			\textbf{2.4} & The data from the sensors will be sent to the companion app within 5 seconds. & Data is received within 5 seconds. 	&	Data is received within 10 seconds	&	Data is received after 10 seconds. \\ \hline  
		\end{tabularx}
		
		\noindent
		%\begin{tabularx}{\textwidth}{| X | X | X |}
		\begin{tabularx}{\textwidth}{|S|B|B|B|B|}
			\hline
			\textbf{Num.}     & \textbf{Criteria}     & \textbf{Desirable}    & \textbf{Acceptable} &  \textbf{Fail}	 \\ \hline
			\textbf{3.1} & The user interface should be clean and simple.        & There is minimal information displayed and no control buttons are visible by default.   & Some control buttons are visible by default but most are hidden.  & All control buttons and elements are visible by default.      \\ \hline
			\textbf{3.2} & The menu will allow full manual control.        & The rover can be switched to full manual control using one control button.	&	The rover can be switched to manual control using the menu.	&  The menu does not allow the rover to be switched to full manual control.       \\ \hline
			\textbf{3.3} & The UI should present the sensor data. & The UI presents graphed sensor data.	&	The UI presents numerical sensor data.	&	The UI does not display sensor data. \\ \hline
			\textbf{3.4} & The Embedded Application will be observable. & Another computer can be used to view the operations of the embedded computer. 	&	A HDMI cable can be used to view operations of the embedded computer. 	&	The operation of the embedded software cannot be observed. \\ \hline   
		\end{tabularx}
		
	\subsection{Stage 2: Input and Output Usability of the Program}
	
		In this section, we will test how useful the input and output of the solution is.\\
	
		\underline{Input:} \\
	
		\noindent
		%\begin{tabularx}{\textwidth}{| X | X | X |}
		\begin{tabularx}{\textwidth}{|X|X|X|X|}
			\hline
			\textbf{Criteria}     & \textbf{Desirable}    & \textbf{Acceptable} &  \textbf{Fail}	 \\ \hline
			The user should be able to control the rover movement using keyboard keys.      &  The user can move the rover in all directions using computer the keyboard.  & The user can move the rover forwards and backwards. & The user has no control over the movement of the rover.      \\ \hline
			The user should be able to interact with hardware control buttons.        & All buttons are functional and available to user	&	Some buttons do not work.	&  No buttons can be clicked.       \\ \hline
			The embedded application should receive input from all sensors on the rover.        & Required sensor information from all sensors can be read and used for algorithms.	&	Some sensor inputs can be read and used.	& No sensor inputs can be read.       \\ \hline
		\end{tabularx}
		\newpage
	
		\underline{Output:} \\
	
		\noindent
		%\begin{tabularx}{\textwidth}{| X | X | X |}
		\begin{tabularx}{\textwidth}{|X|X|X|X|}
			\hline
			% 		        & 	&		&        \\ \hline
			\textbf{Criteria}     & \textbf{Desirable}    & \textbf{Acceptable} &  \textbf{Fail}	 \\ \hline
			 The program outputs the location of the rover.     & The coordinates can be saved to a file with timestamps and location can be displayed  &  The UI displays the rough location of the rover. & The location of the rover can not be found.     \\ \hline
			 The program outputs the information from sensors.       &  All information from all sensors can be presented well (e.g. using graphs).	&	The information can be viewed using the back-end.	&   No sensor information can be found.     \\ \hline
			 The companion program can output the current movement of the rover.        &  The information about the current movement can be saved with timestamps.	&	The information about the current movement can be viewed using back-end.	&  The current movement of the rover cannot be found using companion app.      \\ \hline
			 The embedded application can output signals to move appropriate motors on the rover.       & All motors onboard can be controlled individually by the embedded application.	&	Most motors on the rover can be controlled, enough for movement of the rover.	&   Motor control is not sufficient for movement of the rover.     \\ \hline
		\end{tabularx}
		~\\
	\subsection{Stage 3: Processing Functionality and Robustness}
	
	
	
	
		% Check if appropriate to use. It is kinda long and doubtful. Check with Sir before completing.
		To test how robust the program is, each module of the solution must be tested to ensure that  they all work as required. The lack of strength of one module can be a bottleneck and weaken the whole program, making it unreliable. This test will be a pass or fail test, where each module will either be approved or rejected. Below I will describe the expected output from each module.
		\\\\
		\noindent
		The modules to be tested and how they will be tested:
		\begin{enumerate}
			\item{\textbf{Use the rover's camera system to get some RAW data about the surroundings}}\\\\ \emph{For this step to be rigid, the camera system should be able to give a continuous stream of RAW data to the Raspberry Pi for at least 5 hours.}
			\item{\textbf{Add threshold to data to limit the amount of data being passed through the program}}\\\\ \emph{This is a vital step as it can drastically increase the performance and reliability of the solution. The output of this module should be reduced size image data that displays only the closer items.}
			\item{\textbf{Plot an image representation of the RAW threshold data}}\\\\ \emph{This module will present an image representation of the RAW threshold data on the Raspberry Pi.}
			\item{\textbf{Use image processing to detect the obstacles in the image}}\\\\ \emph{This module will highlight the obstacles in the image.}
			\item{\textbf{Use a smart algorithm to determine the action needed to avoid obstacles}}\\\\ \emph{This module will output the either "right" or "left" or any other verbs to reflect the best course of action to avoid the obstacles when detected.}
			\item{\textbf{Overwrite algorithm-determined movement with manual instructions from companion app if manual mode is enabled.}}\\\\ \emph{If in Manual mode, any instructions from the companion app will replace the instructions from the smart algorithm.}
			\item{\textbf{Send instructions to a driver board (allowing hardware control to motors)}}\\\\ \emph{This module should output electrical signals over the I2C bus between the Raspberry Pi and Arduino(being used as a driver board).}
			\item{\textbf{Use an algorithm on the driver board to control the motors}}\\\\ \emph{The output from this module should be an electrical pulse sent to the DC motors to control movement.}
			\item{\textbf{Send the accurate movement of the rover to the companion app}}\\\\ \emph{The current movement of the rover will be sent to the companion app as a string to represent the direction of movement, but constant velocity will be assumed.}
			\item{\textbf{Calculate vector location of rover using its movement (companion app)}}\\\\ \emph{This module will run on the companion app and will use the current state of movement of the rover to perform vector calculations on the last vector location.}
			\item{\textbf{Get data from the metal detector}}\\\\ \emph{The output of this module will be an integer to represent the strength of the magnetic induction in the metal detector.}
			\item{\textbf{Send metal detector data to companion app}}\\\\ \emph{The output of this module will be a transfer of data from the embedded app to the companion app. This module will output integers in the companion app.}
			\item{\textbf{Write vector location and metal detector data to local file (companion app)}}\\\\ \emph{The output of this module will be a constantly updating file that holds the vector location, metal detector data and timestamps.}
		\end{enumerate}

\section{Limitations}

There are quite a few limitations to this solution. The first is that the solution relies on an active internet connection to transfer data. Although the rover can be kept safe and away from obstacles without an internet connection, the rover needs it to send any data or receive instructions for manual mode. This can be an issue if used for the actual Pluto rover as it would require a satellite to orbit the planet and relay information to Earth, which can be difficult. When used for models and demonstrations, the limitations is that there needs to be an available internet connection that the rover can pair to, which can sometimes be a problem. 
\\\\
\noindent
Another limitation with this solution, is that we do not know the exact details about the terrain and atmosphere of Pluto and we have never sent a rover there, therefore the solution is optimised to our current knowledge of the terrain, which may not be accurate. This means that the solution may not use the optimal algorithm to autonomously navigate the surface of Pluto. 
\\\\
\noindent 
An additional limitation is that if a smart algorithm is used for the autonomous navigation, then it will only respond to pre-programmed circumstances and that may not always be optimal. But, when using a AI algorithm, the training will change the way it will react to unseen circumstances, and so training it on Earth may not make it optimal for Pluto terrain, meaning that it may behave unexpectedly.  
\\\\
\noindent
Another limitations that we have is to do with the available hardware. There isn't powerful hardware available in our context and so the program needs to be extremely efficient. The hardware that we have available cannot run existing solutions such as RTAB-Map and OpenVSLAM to complete the functions that we require (autonomous navigation). This means that a new approach to this problem is needed to make it more efficient and allow it to run on the limited resources available. 
\\\\
\noindent
While there are many existing solutions available that solve problems similar to ours, there are no easily accessible programs that are designed specifically for rovers. This means that there is no available help online that is specific to our context, and there is no example solution from which the basis of the application can be built. This makes developing the solution much more challenging. 

\end{document}
